\section{Introduction}
\label{sec:introduction}

Exploratory Learning Environments (henceforth ELEs) are a particular
type of learning environment where the focus is on students'
exploration of the knowledge domain. Examples of ELE
include simulators, virtual labs, microworlds, and educational
games. ELE give considerable freedom to students, who may explore and
learn in a variety of different ways.

ELE are usually designed to provide opportunities for learners to
develop conceptual rather than just procedural knowledge in the subject
domain. The tasks that students are asked to
undertake are open-ended in nature, may have many alternative
solutions, and encourage students to explore the learning
environment and follow a variety of solution approaches. Research
has found that considerable guidance is required to ensure learning in
such open-ended contexts~\cite{kirschner06,Kynigos92,MayerDiscovery},
but in the presence of adequate support ELE can lead to more engagement
and robust learning (c.f.~\cite{Noss96,JongJoolingen98} and for more
recent reviews of the area~\cite{InquiryLearningJoolingen,Healy2010Charting}. 

In order to provide immediate support to students recent efforts  
(that of our group and others) aim to design intelligent components ... 
that undertake at least some of the trivial tasks of the teacher. 
However, this intelligent support cannot completely replace the teacher 
whose role in exploratory learning is that 
of a 'facilitator’, or
`orchestrator'~\cite{Trouche2004,Hoyles2004Integration}. This role
would be easy to achieve in a one-to-one student-tutor interaction but in already
 busy classroom there are several challenges that are further compounded by the 
use technology. During interaction with ELEs in particular and given the
open-ended nature of the tasks that the students are working on,
 teachers can only be aware of what a small number of
students are doing at any one time as they walk around the classroom –
the computer screens of the students who are not in their immediate
vicinity are not visible to them and those students may not be engaged
in productive construction (going so far, in our own experience, as
browsing the web, playing games, or engaging in online chat). It is
therefore hard for teachers to know which students are currently in
difficulty and in need of their support, which students are making
progress or are off-task. Even for the
students whose screens are currently visible to the teacher, it may be
hard for the teacher to understand the process by which students have
arrived at the current state of their construction and the recent
feedback they have received from the system, and to provide therefore
targeted guidance. 

In this paper we present our approach to design tools that can assist
 teachers in a classroom where students are using ELEs. Our use case
 revolves around an intelligent microworld to that is designed to
 support the developmen of algebraic ways of thinking. We have
 designed and developed a suite of visualisation and notification
tools, which we refer to as the {\em Teacher Assistance (TA)}
tools. The aim of the TA tools is to
assist teachers in focussing her attention across the whole class and to
inform their own interventions to support students in reflecting on
their work, the feedback they have been given by the system, setting
and working towards new goals, and comparing and discussing their
solutions with their peers. 

In~\cite{PearceLazard2010Design,IEEE-TLT-TA} we described the architectural design
and implementation of the TA tools, focussing specifically on one
tool, the Student Tracking (ST) tool (see
Sections~\ref{sec:backgr-relat-work}~and~\ref{sec:teach-assist-tools}~below). 
In contrast, the present paper discusses the pedagogical rationale for
the tools, the teachers’ requirements, and the methodological
challenges we have faced in developing them. We identify the main
usage scenarios of the TA tools and discuss the design and results of
a series of formative and summative evaluation activities, which have
only recently been completed and so were not reported
in~\cite{PearceLazard2010Design,IEEE-TLT-TA}. 
Also, the discussion in the present paper
encompasses the whole suite of MiGen tools targeted at supporting the
teacher in monitoring students’ activities and progress in the
classroom.
 
The outline of the paper is as follows. In
Section~\ref{sec:backgr-relat-work}  we give an
overview of the context and functionalities of the MiGen system, and
of related work in the areas of exploratory learning environments and
support for the teacher. In Section~\ref{sec:methodology}
we discuss the methodology we
have adopted in designing, developing and evaluating the TA tools. We
also discuss the teachers’ requirements from the tools, in the form of
a set of usage scenarios. In Section~\ref{sec:teach-assist-tools}
we describe the tools
themselves - the Classroom Dynamics (CD), Goal Achievements (GA) and
Student Tracking (ST) tools. Section~\ref{sec:formative-evaluation} 
discusses the formative
evaluation of these tools with teachers and teacher educators, and the
changes that were made to the tools as a result of this formative
evaluation. Section~\ref{sec:summative-evaluation} 
presents the results from a set of summative
evaluation activities with the tools, and identifies directions of
further research.  Section~\ref{sec:discussion} 
discusses the outcomes of these
evaluations, and Section~\ref{sec:conclusion} 
gives our concluding remarks and directions
for further research.


%%% Local Variables:
%%% mode: latex
%%% TeX-master: "main"
%%% End:
