\section{Introduction}
\label{sec:introduction}

Exploratory Learning Environments (henceforth ELEs) are a particular
type of learning environment where the focus is on \emph{exploring}
the knowledge domain on the learners' own terms. Examples of ELE
include simulators, virtual labs, microworlds, and educational
games. ELE give considerable freedom to students, who may explore and
learn in a variety of different ways.

ELE are usually designed to provide opportunities for learners to
develop complex 
cognitive skills rather than just knowledge of concepts in the subject
domain [*thompson87,*hoyles93]. The tasks that students are asked to
undertake are open-ended in nature, may have many alternative
solutions, and encourage students to explore the construction
environment and follow a variety of construction approaches. Research
has found that considerable guidance is required to ensure learning in
such open-ended contexts [*kirschner06,*NossHoyles96,*Kynigos92,*Mayer04],
but in the presence of adequate support ELE lead to more engagement
and deeper learning~\cite{deeperLearning}. 

This support is usually provided by human teachers although, as we
will present in Section~\ref{sec:migen-system}, it can be
computer-based support to some extent.
Research in students’ learning with microworlds has highlighted
the indispensable role of the teacher as a ‘competent guide’ (*Leron,
1985), a ‘facilitator’ (*Hoyles, Sutherland, 1989), or ‘orchestrator’
(*Trouche, 2004; *Hoyles et al., 2004) of both well-defined
investigations (*Kynigos, 1992) and goal-oriented exploration, which
aligns with a learning agenda (*Noss \& Hoyles, 1996). However, in
order to achieve those goals effectively for big groups in the
classroom, teachers need to be aware of what is happening. This cannot
happen with the usual tools that teachers have been relying on for
years. They cannot just look at the students and understand what they
are doing, because the students' attention is not focused on the
teacher but on their computers, and the teacher cannot see all
computer screens at the same time (much less if they are
touchscreens). The teacher, therefore, need a way to become aware of
the situation in the classroom if they aspire to become efective
facilitators or orchestrators for their students. 

In this paper we present a microworld-based system that is designed to
be deployed in classrooms within schools ---where a typical class size
is around 30 students---, for supporting the learning of algebraic
generalisation. Given the
open-ended nature of the tasks that the students are working on and
the class sizes, teachers can only be aware of what a small number of
students are doing at any one time as they walk around the classroom –
the computer screens of the students who are not in their immediate
vicinity are not visible to them and those students may not be engaged
in productive construction (going so far, in our own experience, as
browsing the web, playing games, or engaging in online chat). It is
therefore hard for the teacher to know which students are currently in
difficulty and in need of her support, which students are making
progress on the task, and which students may be off-task. Even for the
students whose screens are currently visible to the teacher, it may be
hard for the teacher to understand the process by which students have
arrived at the current state of their construction and the recent
feedback they have received from the system, and to provide therefore
targeted guidance without engaging in a lengthy dialogue with the
student.
 
In our efforts to support teachers in
employing a microworld for algebraic generalisation in the classroom,
we designed and developed a suite of visualisation and notification
tools, which we refer to as the {\em Teacher Assistance (TA)} tools
and which are the focus of this paper. The aim of the TA tools is to
assist the teacher in focussing her attention across the class and to
inform her own interventions to support students in reflecting on
their work, the feedback they have been given by the system, setting
and working towards new goals, and comparing and discussing their
solutions with their peers. 

In~\cite{ectel2010-TA,IEEE-TLT-TA} we described the architectural design
and implementation of the TA tools, focussing specifically on one
tool, the Student Tracking (ST) tool (see
Sections~\ref{sec:backgr-relat-work}~and~\ref{sec:teach-assist-tools}~below). 
In contrast, the present paper discusses the pedagogical rationale for
the tools, the teachers’ requirements, and the methodological
challenges we have faced in developing them. We identify the main
usage scenarios of the TA tools and discuss the design and results of
a series of formative and summative evaluation activities, which have
only recently been completed and so were not reported
in~\cite{ectel2010-TA,IEEE-TLT-TA}. 
Also, the discussion in the present paper
encompasses the whole suite of MiGen tools targeted at supporting the
teacher in monitoring students’ activities and progress in the
classroom, not just the ST tool.  Our focus in this paper is on tasks
undertaken individually by students. MiGen also provides a tool to
support the teacher in grouping students for the collaborative tasks
that follow on from their individual constructions~\cite{GroupingTool}
and providing further support for
students’ collaboration is an area of on-going research.
 
The outline of the paper is as follows. In
Section~\ref{sec:backgr-relat-work}  we give an
overview of the context and functionalities of the MiGen system, and
of related work in the areas of exploratory learning environments and
support for the teacher. In Section~\ref{sec:methodology}
we discuss the methodology we
have adopted in designing, developing and evaluating the TA tools. We
also discuss the teachers’ requirements from the tools, in the form of
a set of usage scenarios. In Section~\ref{sec:teach-assist-tools}
we describe the tools
themselves - the Classroom Dynamics (CD), Goal Achievements (GA) and
Student Tracking (ST) tools. Section~\ref{sec:formative-evaluation} 
discusses the formative
evaluation of these tools with teachers and teacher educators, and the
changes that were made to the tools as a result of this formative
evaluation. Section~\ref{sec:summative-evaluation} 
presents the results from a set of summative
evaluation activities with the tools, and identifies directions of
further research.  Section~\ref{sec:discussion} 
discusses the outcomes of these
evaluations, and Section~\ref{sec:conclusion} 
gives our concluding remarks and directions
for further research.




%%% Local Variables:
%%% mode: latex
%%% TeX-master: "main"
%%% End:
