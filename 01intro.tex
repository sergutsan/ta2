\section{Introduction}
\label{sec:introduction}

Exploratory Learning Environments (ELEs) are a particular
type of learning environment where the focus is on students'
exploration of the knowledge domain. Examples of ELEs
include simulators, virtual labs, microworlds, and educational
games. ELEs give considerable freedom to students, who may explore and
learn in a variety of different ways.
%
% ELEs are usually designed to provide opportunities for learners to
% develop conceptual rather than just procedural knowledge in the subject
% domain. 
%
The tasks that students are asked to
undertake are open-ended in nature, may have many alternative
solutions, and encourage students to explore the learning
environment and to follow a variety of solution approaches. Research
has found that considerable guidance is required to ensure learning in
such open-ended contexts~\cite{kirschner06,Kynigos92,MayerDiscovery},
but that in the presence of adequate support ELEs can lead to more engagement
and deeper learning (see~\cite{Noss96,JongJoolingen98} and, for more
recent reviews of the area,~\cite{InquiryLearningJoolingen,Healy2010Charting}). 

In order to provide immediate support to students as they are interacting
with an ELE, recent efforts aim to design intelligent components that 
undertake at least some of the simple aspects of providing feedback to 
students~\cite{MiGen-JRPIT,AmersiConati09}. 
However, this intelligent support cannot completely replace the teacher 
whose role in an exploratory learning setting is that 
of a `facilitator', or `orchestrator'~\cite{Trouche2004,Hoyles2004Integration}. 
This role would be relatively easy in one-to-one student-tutor 
interaction, but scaling it up to the number of students in a 
typical classroom poses several challenges,
that are further compounded by the use technology. 
Given the open-ended nature of the tasks that the students are working on,
teachers can only be aware of what a small number of
students are doing at any one time as they walk around the classroom. 
The computer screens of students who are not in their immediate
vicinity are typically not visible to them and these students may not be engaged
in productive construction (going so far, in our own experience, as
browsing the web, playing games, or engaging in online chat). It is
therefore hard for teachers to know which students are making progress,
which are off-task, and which are in difficulty and in need of additional support.
Even for thse students whose screens are currently visible to the teacher, 
it may be hard for the teacher to understand the process by which students have
arrived at the current state of their construction and the recent
feedback they have received from the system, and to provide appropriate
guidance. 

In this paper we present our approach to designing tools that can assist
teachers in a classroom where students are using an ELE. 
Our case study is an intelligent microworld designed to
support 11-14 year old students' development of algebraic ways of thinking. 
We have designed a suite of visualisation and notification
tools, which we refer to as the {\em Teacher Assistance (TA)}
tools. The aim of these tools is to assist teachers in focussing their 
attention across the whole class as students are working with the microworld,
and to inform teachers' own interventions in supporting students to reflect on
their work, on the feedback given them by the microworld and in setting
and working towards new goals. 
% Note from AP: the below is not discussed in this paper, and is more the focus
% of the grouping tool I think:
%,   and in comparing and discussing their
%    solutions with their peers. 

In~\cite{PearceLazard2010Design,IEEE-TLT-TA} we described the architectural design
and implementation of the TA tools, focussing specifically on 
one tool, the Student Tracking (ST) tool (see
Sections~\ref{sec:backgr-relat-work}~and~\ref{sec:teach-assist-tools}~below). 
In contrast, the present paper discusses the pedagogical rationale for
the TA tools, the teachers' requirements from such tools, and the methodological
approaches we have followed in developing them. We identify the main
usage scenarios of the TA tools and discuss the design and results of
a series of formative and summative evaluation activities (these have
only recently been completed and so were not reported
in~\cite{PearceLazard2010Design,IEEE-TLT-TA}).  
Also, the discussion in the present paper encompasses the whole suite of 
TA tools targeted at supporting the teacher in monitoring 
students' activities and progress in the classroom.
 
The outline of the paper is as follows. In
Section~\ref{sec:backgr-relat-work}  we give an
overview of the context and functionalities of our system, and
of related work in the areas of exploratory learning environments and
support for the teacher. In Section~\ref{sec:methodology}
we discuss the methodology we
have adopted in designing, developing and evaluating the TA tools. We
also discuss teachers' requirements from the tools, in the form of
a set of usage scenarios. In Section~\ref{sec:teach-assist-tools}
we describe the tools
themselves --- the Classroom Dynamics (CD), Goal Achievements (GA) and
Student Tracking (ST) tools. Section~\ref{sec:formative-evaluation} 
discusses the formative evaluation of the tools with teachers and 
teacher educators, and changes that were made to them as a result.
Section~\ref{sec:summative-evaluation} 
presents the results from a set of summative
evaluation activities with the tools.
Section~\ref{sec:discussion} discusses the outcomes of these
evaluations. 
Section~\ref{sec:conclusion} 
gives our concluding remarks and directions
for further research.


%%% Local Variables:
%%% mode: latex
%%% TeX-master: "main"
%%% End:
