
\section{Formative Evaluation}
\label{sec:formative-evaluation}

During Phase C of the project, formative evaluation of the ST, CD and GA tools was undertaken
with respect to the usage scenarios identified from Phase B. 
This formative evaluation comprised two parts.

The first part was a 3-hour evaluation session undertaken with 26
trainee Maths teachers on the Postgraduate Certificate in Education
programme at the Institute of Education, University of London, who were split into two
parallel groups of 13 for logistical reasons. Each of the participants
had an installation of the MiGen system running on their computer. In
the first half of the session, participants were introduced to the
MiGen system as a whole and to the eXpresser tool. 
Participants were then asked to work through several
construction examples using the eXpresser so as to gain familiarity
with how students might use it in a lesson and the kinds of feedback
the system would give to students. There was then a 15 minute
break. In the following 30 minutes, each of the TA tools was
introduced to the participants, using real data drawn from one of the
classroom trials undertaken in Phase B. 
Using the time-stop functionality, the research team `froze' the
display of the data at a time 10 minutes into the lesson and explained
to the participants the information that was being shown in each tool. 
For the final hour of the session, participants were asked to move the display 
of the TA tools on their computer forwards, firstly to 30 minutes into the lesson, and
then to 5 minutes prior to the end of the lesson. For each of these two
time points, participants were asked to answer a short list of 
questions relating to usage scenarios US1--US6.
At the end of the session, participants were asked to complete a similar 
questionnaire, this time relating to the full set of usage scenarios US1--US8. 
For each question, we also asked participants how they would tackle the corresponding
usage scenario if the TA tools were not available to them. 
% The aim of these
% questionnaires was to elicit feedback relating to the extent to which
% the TA tools met the requirements of the usage scenarios.   
Two participants did not complete the questionnaires.  
We summarise below the answers of the remaining 24 participants 
with respect to each of the usage scenarios.

\noindent{\textit{US 1. Finding out which students need the teacher's immediate help}}
 
For both time points (30 minutes into the lesson and 5 minutes before
the end), 23 participants answered that they would use the
CD tool and look for students whose circles were coloured red. 
Most responses for the first time point
related to `keeping an eye' on those students 
who did not seem to have had a good start and
ensuring that those ``marked with amber are staying focused''.
For the second time point, providing help to students seemed more important.
Two teachers mentioned explicitly the
functionality of clicking on the students' circles to see what the students have
done, and if any are falling behinding grouping them to help them collectively. 
Ten respondents gave detailed comments such as: ``visual review of everyone's status
helps you really check if students are understanding, without asking
each of them individually'', ``allows much faster response to students' queries'', 
``clearly highlights who is active and who isn't'', 
``also shows what they have been doing which helps as sometimes,
although the student is active, they may not be on task''. 
Two responses went beyond using the CD tool and recognised
that ``GA also shows levels of progress'' and can be used also for
deciding which students need help. 
 
With regards to finding out which students need immediate help without
using the TA tools, 14 participants answered that they
would use a traditional classroom solution, e.g. ``walk around the
classroom viewing students' work and helping any student who may be
off task or needing help'', or ```students would have to put their hand up
for my immediate attention''. Eight came up with more innovative
solutions like using `traffic light' cards to allow students to indicate if
they are doing well (green card) or might be in trouble soon (orange
card) or definitely need help now (red card), or even a remote
control that would allow students to `call the teacher'. 

% In relation to doing the same task with the tools, 10 out of the 24
% demonstrated a deep appreciation of the tools stating that they would
% use a combination of the tools starting from the CD tool but also
% giving elaborate comments such as ``visual review of everyone's'' status
% helps you really check if students are understanding, without asking
% each of them individually'' ``allows much faster response to students'
% queries'' or ``It clearly highlights who is active and who isn't'', ``it
% also shows what they have been doing which helps as sometimes,
% although the student is active, they may not be on task'' showing that
% beyond the obvious answer of relying on the visualisation of the CD
% tool the combination of the tools can provide much more elaborate
% information to help decision making. 8 out of 24 provided
% straightforward answers such as ``simple monitoring such as Class
% Dynamics and Goal Achievement give a quick overview'' still often
% demonstrating that they appreciate the potential of combining
% information from different tools. 4 teachers did not answer clearly
% but 2 provided simple UI suggestions with the main one (also mentioned
% from other teachers especially when providing verbal feedback) being
% the suggestion for using the colours in the CD tool to also indicate
% goal achievement as additional information that could help in
% prioritizing which students need help. 
 

\noindent{\textit{US2. Finding out which students are progressing satisfactorily towards completing the task and which ones may be in difficulty.}
 
For this usage scenario, for the first time point, 19 participants referred 
to the GA tool, e.g. ``students who have yet to achieve any goals are in
difficulty'' or to a combination of GA tool with the CD and ST tools.  
Another five mentioned only the CD tool as a means of finding out in a
glance which students are progressing. 
For the second time point, most referred to their previous
answer or did not provide an answer. 
Two of the five who did answer
said that they would use the CD tool, and one commented that it
would be useful to have the ability to see this kind of information on 
a per task basis.  
 
Without use of the TA tools, 23 participants stated again 
that they would use a traditional classroom approach, e.g. ``periodically
ask whole class re. stage of progress, level of understanding, plus
constantly circulate to observe them at work and assist as
required''. Two of these responses contained a comment about the effort
that this would require. One teacher came up with a more innovative
approach: ``I could get the pupils to personally fill in a tick box of
how they have progressed, though this would take extra time and effort''. 

% In relation to doing the same task with the tools, 13 students
% seemed to appreciate how the “tools give you [..] a ‘screenshot’ of
% what they have produced.”


\noindent{\textit{US3: Finding out which students are currently disengaged from the task.}


  For this usage scenario, all 24 participants provided at least one
  answer to either of the time points demonstrating an appreciation to
  the fact that they would look for circles coloured Amber in the CD
  tool and provided answers such as ``I would be most concered about
  amber students who had yet to achieve any tasks''. Again the
  difference between the two time points related to the level of
  intervention they would undertake.  Some of them recognised that
  towards the end of the lesson students could be disengaged because
  they had finished with the task and therefore one approach could be
  to get ``those who have finished to help those who are still
  struggling''.

Some comments revolved around
teachers' need to be able to configure the length of the time period 
that would cause a circle to be coloured Amber
(this is an option that we subsequently added to the CD tool), 
and also ideas about the type of
disengagement that may be ocurring: it is different for a student to, for example, 
have the eXpresser open and be thinking about the task set or be discussing 
it with other students, and having the eXpresser minimised and playing 
a game or browsing the internet. 
%
% Four participants did not answer but provided UI-related and other comments. 
% Most of these related to the fact that the disengagement information can be
% misleading e.g. ``Might highlight pupils who are just having a break to
% think [or taking notes]. Also some pupils might be active but just
% making up patterns on the program i.e. not sticking to task'' 
%
One commented that students may be able to ``game the system'' as they 
could ``figure out how to avoid amber by moving the mouse or dragging and 
dropping every n minutes''.   
Some teachers therefore commented that they would have
to observe the students' working as well, and some of them 
wondered whether it would be possible to see 
the students' screens on the teacher's computer. 
As there is existing software that can achieve this, we have not focused 
providing such functionality in our TA tools. 
Our view is that the disengagement information that is provided by the CD tool 
is a first sign for the teacher to do exactly what the 
participants mentioned, which is to approach the students coloured amber 
to find out who is actually disengaged from the task; 
or to use the information to choose which students' screens to observe
using additional off-the-shelf software. 

Without use of the TA tools, 22 participants
answered that they would use again some traditional
classroom solution.
% AP note : this is a repeat of the earlier comment: 
%  e.g. “Periodically ask whole class re stage of
% progress, level of understanding, Plus general observation… and
% constantly circulate to observe them at work and assist as
% required..“. 
Three commented on the difficulty of doing this, e.g. 
``Difficult to keep an eye on all pupils, especially when you are
helping one pupil in particular. Pupils often see you coming and
switch back to task as you approach''. 
Fifteen participants provided answers that demonstrated their appreciation of the 
TA tools in overcoming these difficulties, e.g. ``The tools give you a clear
indication of who is currently inactive or has stopped working on
their task for over a minute. This gives you a clear indication of who
might be switching off''. 
Five provided elaborate comments
demonstrating an appreciation also of the transformative nature
of the information provided, e.g. ``The tools provide a way of helping to
increase the efficiency of my role as a teacher. By enabling me to
look at the student tracking tools and the goal achievement tools I
can readily identify those pupils that appear to be disengaged and
subsequently target them for additional support/encouragement''. 



\noindent{\textit{US4: Identifying common conceptual and procedural difficulties
students are facing in order to provide more explanation to the class
as a whole.}  

For this usage scenario, we asked the questions ``Would this be
a time-point that you would give more explanation to the class as a
whole?  And if so what would you say based on information from the tools?''. 
From the 17 participants who answered, 9 referred
to the GA tool and to the white cells that show lack of achievement
of task goals;  
4 would rely on the ST tool, but commented on its complexity
for use during the lesson; 
2 mentioned the CD tool but without providing any explanation; 
and 2 said that they would re-cap the task set for the whole class, 
and perhaps ask one of the students to demonstrate their solution,
but did not explain why. 

Once again, the difference between the two time points was the level of teachers' 
intervention: at the first time point, 
answers revolved around making sure that students were progressing 
whereas at the second time point participants were concerned with ensuring 
that students have achieved important objectives and with wrapping up the lesson.
% 
% Apart from commenting on the complexity of the ST
% tool, 
% Other UI improvements involved the scrolling needed for goal
% achievement.

% Without use of the TA tools, ??? MANOLIS please write a couple of sentences 
% about students' responses for this use case without the TA tools 

Without the use of the TA tools, 15 particants provided comments, most
revolving around walking around the class and either observing or
asking students of different abilities to explain what
they have done. Two particants said that they would generate a whole
class discussion referring to critical instances that they would have
noticed walking around the class. Another two participants would 
employ some explicit (formative) assessment in order to ascertain the
level of the class  but also commented on the difficulty of the approach.

\noindent{\textit{US5: Finding out which students have finished the task}

For this usage scenario we asked participants to give examples
of students who had finished the task, as well as any additional comments
they might have. 
21 participants provided correct answers for both time points. 
13 of these cited using the CD tool (the annotations of the
number of goals achieved within the students' circles) 
and the rest cited use of the GA tool. 
Five participants commented on the ease
with which they could check which students have finished the task. 
One participant said that close to the end of the lesson they would
consider displaying GA information on the interactive
whiteboard for all students to see and encourage the ones that 
that are behind to catch up. 
Three participants provided additional
ideas on how they would take advantage of the GA tool: 
it could help them identify difficult tasks that they may need to modify, 
or allow them to choose which students to give extension tasks to. 

Without use of the TA tools, all 24 respondents acknowledged that they 
would have to revert to a traditional approach whereby they would ask 
students who have finished the task to raise their hands.  

% 15 out of the 24 answered the related questions in the end of session
% questionnaire. They  provided answers that demonstrated their
% appreciation of the tools. 3/24 participants even provided additional
% ideas on how they would take advantage of the GA tool. In particular,
% according to their comments, it could help them identify difficult
% tasks that they may need to modify or allow them to choose which
% students to give extension tasks to. With respect to limitations, the
% main requests were for the tool to  take into account ``the number of
% activities a student has gone through''  (probably meaning in a
% particular session in order to weight the information provided) and
% that students who have completed their task are also somehow displayed
% in the CD tool to allow for quick identification and action on behalf
% of the teacher (e.g. to be given extension work or to help other
% students) 


\noindent{\textit{US6: Finding out which students have achieved which task goals.}

The results from this question were very similar to those for US5 above. 
Many participants replied ``see above'' or commented on the intuitive
and simple usage of the GA tool. 

% The only comment worth reporting from
% the end of session questionnaire was one participant's confusion
% about the three degrees of achievement (i.e. the fact that a goal can
% be retracted). We think that was an outlier or that maybe he missed
% the corresponding explanation as no one else commented on that either
% in this session or in subsequent ones. Neverthless, this comment acts
% as a reminder for us to ensure that this point is sufficiently
% explained in instructions but also that the mouse over information is
% descriptive enough.   

\noindent{\textit{US7: Providing appropriate support and guidance to individual students
(i) during the lesson, and (ii) after the lesson.} and \textit{US8: Reflecting on the class' achievements and planning the next lesson.} 

As these are more open-ended use cases, we asked participants to
answer questions relating to them only in the end-of-session questionnaires. 
All 14 participants who provided answers showed 
an appreciation of the functionalities of the TA tools and 
how they can be combined to allow teachers to provide support 
to students during and after the lesson. 
Five stated that they would look back at the TA tools 
after the class to check which task goals were being accomplished 
and which were problematic. 
This would allow them to find out where and how students were struggling.
% (one of
% them commented that this information could be taken into account for
% the grouping).
Two commented that the CD tool could help in making sure
that students who need immediate help during the lesson are 
supported, and that for those students who had not received enough 
help it would allow the teacher to plan to talk to them in the next lesson.
One commented that having access to the TA tools 
``could be very helpful after the lesson to assess progress and
decide which pupils need more support next lesson,
and which pupils need stretching further''. 
Two commented that seeing the
indicators in the ST tool allows teachers
``to identify the most common misconceptions which could then be
consolidated in the following lesson'', although both referred to the
complexity of the ST tool for use during the lesson.

Two participants also commented positively on the eXpresser's feedback 
provided to students, which although designed to support students
and not presented as part of the TA tool evaluation has of course
the potential to assist teachers by reducing their workload in supporting
the class.
%
% \ednote{write about' ``I wonder how much it indicates to the teacher 
% regarding the area or the level of this problem or 
% (ii) The tool suggest how much work students were able to complete''
% and `` it would be difficult to digest all the information the tools
% provide during the lesson'' 
% and ``Could be very helpful after the lesson to assess progress and
% decide which pupils need more support next lesson,
% and which pupils need stretching further''}

Without use of the TA tools, nine participants provided comments 
that implied that they would adopt approaches requiring them
to either walk around the class and check all students' work or to rely on
students asking for assistance. They commented on the effort that this
would require and the fact that it would be difficult to check students'
work after the lesson class (the teacher would need to organise where students 
save their work so that the teacher could access it). 
Three participants said that they would rely on peer interaction and direct 
discussions between the students, and two referred to more innovative solutions 
such as ``traffic lights'' (similar to the answers for US 1).
 
% Ten participants did not reply and five referred to the answers they gave in US 1.


\bigskip 

The second part of the formative evaluation comprised a focus group meeting
held with a group of pedagogical experts in maths education, to obtain
detailed feedback to inform the development of the final versions of
the tools in preparation for the summative evaluation in Phase D.
The feedback resulting from this focus group meeting comprised
%
%\ednote{I think the details here are not needed and perhaps
%  is just better to say something general like ``The feedback resulting
%  from this focus group meeting in addition to specific comments from
%  the first part with the trainee maths teachers resulted to changes
% prior to the summative evaluation ...'' and perhaps only mention
% about the subset of 'important' indicators as the main change}
%
(i) some change requests relating to the visualisations of all three tools,
%
%     
%  recommendation that all information relating to `state'
%  indicators to be turned off, by default, in the ST tool.
%  NOTE FROM AP - this second point is subsumed by the nex point below:
%
and (ii) the identification of a subset of `important' indicators that 
are most relevant for use by the teacher during the lesson and that should 
be displayed by default in the ST tool. 



%%% Local Variables:
%%% mode: latex
%%% TeX-master: "main"
%%% End:
