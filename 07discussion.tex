
\section{Discussion}
\label{sec:discussion}

The results from both the Formative and the Summative evaluations of
the TA tools are quite positive. From the analysis of the results of
the formative evaluation session held with trainee maths teachers in
Section 5, we can see that participants understood the capabilities of
the tools and were able to use them effectively in answering the
usage-scenario based questions. 

From the analysis of the results of the classroom trial held with a
teacher in her school as part of the Summative evaluation in Section
6, we observed that the teacher was able to use the full range of TA
tools to address the full range of usage scenarios as she undertook a
lesson using MiGen. In the subsequent lesson, without the TA tools,
she ...

*** SERGIO:  full details are needed here, summarising the earlier
analysis in Section 6: The teacher provided very positive feedback
... was really engaged during the class session …, 
and later reported to really miss the use of the tools. ... ***

The analysis of the results of the summative evaluation session
were also satisfactory both because the questions 
were perceived overall as not taking too much time to answer 
with the TA tools and because participants overall agreed with the
statements about the tools' usefulness. 

All of these outcomes point to the usefulness of the TA
tools. None-the-less, the limited number of real classroom trials that
it has been possible to undertake given the timescale and resources of
the project, and the difficulties that some evaluation participants
faced in using the tools for the more complex usage scenarios of
formulating their guidance to students and identifying common
difficulties that students were facing, would point to the need for
further research, both in attempting to further elicit teachers’ needs
from such tools and in undertaking more classroom-based
trials.\ednote{I think this is not needed at all (neither does the
  whole section here). Most of these could move/merged with the discussion}

We have found that undertaking research in the use of exploratory
learning environments (ELE) in schools and, specifically, research in
tools for assisting the teacher is extremely challenging in the
UK. There are two main reasons for this. First, ELE are not a normal
part of classroom teaching most schools we have collaborated with. It
is our impression that the reason for this relates to the need for
significant support from the teacher in order for ELE to be effective
in the learning process. As tools such as MiGen that provide
intelligent support to students as they are undertaking exploratory
tasks become more common, this situation may change, and teachers may
be more willing to include use of ELE in their lessons. A second
challenge we faced was that the use of Teacher Assistance tools such
as those that MiGen provides is quite new to teachers, and requires a
change from their usual routine of walking around the classroom to see
how students are progressing and to provide help on the basis of their
observations. On the positive side, we have seen evidence in our
formative and summative evaluations with current and future maths
teachers that tools such as the ones we have presented here may be
playing a role in reversing this situation.
 
As described in Section 6, the teacher who used our TA tools in the
last classroom trial was “extremely pleased” with them. One of the
main reasons seemed to be the experience of control over the class
that she was able to gain using them. With a quick glance at what the
TA tools were displaying, this teacher was able to know which students
were making good progress, which students were waiting for her help,
and which students were falling behind with respect to the task goals.
 
We believe that one of the main factors in the resistance to the use
of ELE in the classroom is the teacher’s sensation of lack of control
of the situation: traditional learning is structured in nature, with
the teacher being in control of the pace and direction of the
students’ learning; in contrast, exploratory environments challenge
these assumptions. Therefore, tools that can empower teachers by
making them aware of the `state’ of their classroom may serve as a
bridge that facilitates increased use of ELE in classrooms in the
future.
 
We believe that visualisation and notification tools that are
developed specifically to support the teacher in using ELE in the
classroom, such as MiGen’s TA tools, are better able to provide this
sense of awareness for the teacher than are general-purpose screen
monitoring tools that could also monitor and present the students’
actions to the teacher. Screen monitoring tools such as
~\cite{monitor1,monitor2,monitor3,monitor4,monitor5}) are not designed
for continuous intensive observation of the actions of many students
during an entire lesson. Such tools would require a significant effort
on the part of the teacher to follow and analyse students’ actions
comprising clicks, opening and closing of windows, etc. Moreover, such
tools require large screens to be really useful, which is not feasible
during a typical lesson, where the teacher generally needs to be able
to walk round the classroom helping students, as well as using the
tools to monitor the overall classroom state. Although teachers
usually have a dedicated desk-top computer and often a whiteboard in
the classroom, our studies in classrooms have shown that teachers
prefer to interact with the TA tools on a portable device that they
can carry with them as they walk around the class, rather than being
required to walk back to their desk at the front of the class in order
to view a tool, change the display, focus in on one aspect etc. Our TA
tools have been tested and used by teachers on tablet PCs, and it
would be straightforward to adapt them to an even smaller screen such
as that of a smartphone. This is an area of future work.
 
One aspect that is worth mentioning is the effect that this kind of
\emph{teacher awareness} has on the students. In the latest classroom
trial, students reacted strongly when the teacher announced that she
was able to observe from her tablet PC what they were doing on their
computers (some of them quite vocally). It was apparent that the
sensation of being monitored by the teacher led to better general
behaviour and more focused work on the part of the students. We cannot
tell if this "better behaviour under vigilance" effect is something
that would last if the TA tools were used on a daily basis, or whether
students would soon return to their old habits. We believe, however,
that the effect could be a lasting one if students realised that their
actions could have consequences, e.g. if they did not finish most of
the task set in the lesson, the teacher would know this and take it
into account when assessing their overall performance.
 
Finally, it is important to note that the TA tools presented in this
paper are general in their design and could in principle be used to
monitor the activities of students interacting with any ELE as long as
the environment complied with two main requirements: (i) appropriate
interaction indicators are generated and are sent to a server computer
that the TA tools can retrieve them from, and (ii) these indicators
include at least a minimum set of indicators relating to students’
current activity status, waiting for help from the teacher, and goal
accomplishment status, which are the indicators that drive the CD and
GA tool visualisations which we have found, in practice, that teachers
consult most often during a lesson. In the future, we plan to
investigate the integration of our TA tools with other projects that
use indicator-enabled ELE in the classroom.\ednote{Mention here
  Metafora and perhaps cite Silas work and/or our future plans }