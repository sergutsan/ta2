\section{Discussion}
\label{sec:discussion}

The results from both the Formative and the Summative evaluations of
the TA tools are encouraging. From the analysis in Section 5 of the results 
of the formative evaluation session held with a cohort of trainee Maths teachers,
we see that participants understood the capabilities of
the tools and were able to use them effectively in answering most
of the usage-scenario based questions. 
The number of responses fell in relation to usage scenarios US4 and US7, 
that concern identifying common conceptual and procedural difficulties
students are facing, 
providing more explanation to the class as a whole, and 
providing guidance to individual students during and after the lesson.
This finding is corroborated by the 
classroom trial held with a teacher at her school, discussed in Section 6,
who reported that answering such questions
required having a global view of the class's `learning status' 
that was difficult to obtain from the tools.
None-the-less, we were pleased to see that a small number of the 
trainee Maths teachers commented that being able to view the occurrence 
of all the indicators in the ST tool does allow teachers
``to identify the most common misconceptions which could then be
consolidated in the following lesson''. 


From the analysis in Section 6 of the results of the classroom trial held with a
teacher in her school as part of the Summative evaluation,
we observed that the teacher was able to use the full suite of TA
tools to address usage scenarios US1, US2, US3, US5, US6, US8 as she undertook a
lesson using MiGen. As noted above, she had difficulty with US4 and US7.
None-the-less, she reported being ``extremely pleased'' with the TA tools. 
One of the main reasons for this seemed to be the experience of control over the class
that she was able to gain using them. With a quick glance at the information being
displayed by the TA tools, this teacher was able to know which students
were making good progress, which students were waiting for her help,
and which students were falling behind with respect to the task goals.
After the subsequent lesson, undertaken without using the TA tools,
she reported that it was not possible to obtain a view
of the class' progress for so many (28) students, 
and that having access to the TA tools during the first lesson 
had made a real difference. 

From the analysis in Section 6 of the second part of the Summative
evaluation, held with another cohort of trainee Maths teachers, 
we see that participants were able to use the tools to provide
correct answers to questions relating to US1--US6 and that no
answers were perceived as requiring ``a long time'' or ``a lot of time''.
In the analysis of the answers to the end-of-session questionnaire,
relating to participants' perceived usefulness of the TA tools, 
only Q7 (relating to usage scenario US4) had any answers of Disagree 
(2 such answers, out of 11 participants); Q7 also had 4 answers of Not Sure.
Q4 (relating to US2 and US7(i)) also attracted 4 answers of Not Sure. 


All of these results point to the usefulness of the TA tools for the
identified usage scenarios. 
None-the-less, the limited number of classroom trials that
it has been possible to undertake given the timescale and resources of
the project, and the difficulties that some evaluation participants
faced in using the tools for the more complex usage scenarios of
identifying common difficulties that students were facing and
formulating their guidance to students, point to the need for
further research, both in attempting to further elicit teachers' needs
from such tools and in undertaking more classroom-based trials.


% We have found that undertaking research in the use of exploratory
% learning environments (ELE) in schools and, specifically, research in
% tools for assisting the teacher is extremely challenging in the
% UK. There are two main reasons for this. First, ELE are not a normal
% part of classroom teaching most schools we have collaborated with. It
% is our impression that the reason for this relates to the need for
% significant support from the teacher in order for ELE to be effective
% in the learning process. As tools such as MiGen that provide
% intelligent support to students as they are undertaking exploratory
% tasks become more common, this situation may change, and teachers may
% be more willing to include use of ELE in their lessons. A second
% challenge we faced was that the use of Teacher Assistance tools such
% as those that MiGen provides is quite new to teachers, and requires a
% change from their usual routine of walking around the classroom to see
% how students are progressing and to provide help on the basis of their
% observations. On the positive side, we have seen evidence in our
% formative and summative evaluations with current and future maths
% teachers that tools such as the ones we have presented here may be
% playing a role in reversing this situation.
% 
% We believe that one of the main factors in the resistance to the use
% of ELE in the classroom is the teacher’s sensation of lack of control
% of the situation: traditional learning is structured in nature, with
% the teacher being in control of the pace and direction of the
% students' learning; in contrast, exploratory environments challenge
% these assumptions. Therefore, tools that can empower teachers by
% making them aware of the `state' of their classroom may serve as a
% bridge that facilitates increased use of ELE in classrooms in the
% future.
% 
 
In the latest classroom trial, students reacted strongly when the teacher 
told them that she was able to observe from her tablet PC what they were 
doing on their computers. % (some of them quite vocally). 
It was apparent that the sensation of being monitored by the teacher 
led to better general behaviour and more focused work on the part of 
the students. 
We cannot tell without further empirical studies if this 
`better behaviour under vigilance' effect 
would last if the TA tools were used on a regular basis, or whether
students would soon return to their old habits. 
We believe that the effect could be a lasting one if students realised that their
actions would have consequences, e.g. if they did not finish
the task set in the lesson, the teacher would assign them finishing the task
as additional homework, or would take non-completion into account in assessing
their overall performance on this topic. 

We believe that visualisation and notification tools such as MiGen's TA tools 
that are developed specifically to support the teacher in an exploratory learning setting 
in the classroom are better able to provide a sense of awareness for the teacher 
than are general-purpose screen monitoring tools.
% ~\cite{monitor1,monitor2,monitor3,monitor4,monitor5}) 
Screen monitoring tools are not designed for continuous observation 
of the actions of many students during an entire lesson. 
Such tools would require a significant effort
on the part of the teacher to follow and analyse students' actions
comprising clicks, opening and closing of windows, etc. Moreover, such
tools require large screens to be really useful, which is not feasible
during a typical lesson, where the teacher generally needs to be able
to walk round the classroom helping students in addition to using the
tools to monitor the overall classroom state. 
% Although teachers
% usually have a dedicated desk-top computer and often a whiteboard in
% the classroom, our studies in classrooms have shown that teachers
% prefer to interact with the TA tools on a portable device that they
% can carry with them as they walk around the class, rather than being
% required to walk back to their desk at the front of the class in order
% to view a tool, change the tool selected for display, focus in on one aspect etc. 
