\subsection{Related Work}
\label{sec:related}

To our knowledge, MiGen's TA tools (preliminary results were published
in~\cite{TA-ECTEL}) represent the first work that is targeted at
notifying teachers of students’ progress and state during constructionist
learning activities in the classroom, notifying the teacher of
students’ attainment of key indicators, and aiming to inform the
teacher’s own interventions in the class. This novelty of our TA tools
has presented a number of methodological challenges, which we discuss
in Section 3. In the last two years, several similar initiatives have
appeared, including the work at the Metafora
project~\cite{Dragon13}\ednote{SG: I think there was a better paper,
  but I cannot find it now. Maybe MM can help.}, although their
feedback focuses on the statistics of the interaction (e.g.~how often
did the student produce a certain kind of indicator, 
c.f.~\cite{Gueraud09}). As explained in
this paper, we believe that immediate on-line feedback is
more valuable to teachers for orchestrating the use of technologies in
the classroom, an approach recently shared by~\cite{Gutierrez12}
(inspired by early work by~\cite{Yardi08}). Gutierrez focuses on
progress measurement and whether students are stuck in the context of
programming labs. 

The trend towards teacher support appears to grow
also in the learning analytics community
(see for example~\cite{Crespo12,Zaldivar12,Pardo12}, and there is high
synergetic potential between their work and the work reported here.
Other related initiatives include 
using Web log data generated by course management systems
(e.g.~WebCT) to help instructors become aware of students’
activities in distance learning classes~\cite{Mazza07}; post-analysis of
the system's data logs for helping teachers understand students'
behaviour in adaptive tutorials~\cite{BenAnim08}; or providing
awareness information to teachers so as to support their role as 
moderators of multiple e-discussions~\cite{Wichmann09}; although none of them
focuses on exploratory learning activities or environments. 
Particulartly interesting is the work described in 
is~\cite{Avouris08}, that uses tools to analyse CSCL 
synchronous interaction to help the teacher; their use of ``rules'' to
find specific landmarks in the interaction bears some
similarity to the search of indicators we report here. 


% Stefan Weinbrenner, Jan Engler, Astrid Wichmann, Ulrich Hoppe
% Monitoring and Analyzing Students' Systematic Behaviour by Agents -
% The SCY Pedagogical Agent Framework. Proceedings of the 5th European
% Conference on Technology Enhanced Learning (EC-TEL 2010), Barcelona,
% ES, 2010

% -- this I think is one of those mentioned at the top
% Stefan Weinbrenner, Jan Engler, Astrid Wichmann, Ulrich Hoppe
% Monitoring and Analyzing Students' Systematic Behaviour by Agents -
% The SCY Pedagogical Agent Framework. Proceedings of the 5th European
% Conference on Technology Enhanced Learning (EC-TEL 2010), Barcelona,
% ES, 2010


% Discussion here also of other related work on: 
% \begin{itemize}
% \item  interaction analysis
% \item  work on CSCL (even if we are not targetting collaboration here)
%   to support teachers 
% \item  work on teachers’ tools for moderating online discussions and
%   argumentation 
% \item  other tools for learning
% \item Argonaut's teacher tools (Astrid Wichman, Ulrich Hoppe)
% \end{itemize}


%%% Local Variables:
%%% mode: latex
%%% TeX-master: "main"