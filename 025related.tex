\subsection{Related Work}
\label{sec:related}

To our knowledge, MiGen's TA tools represent the first work ---
preliminary results were published in~\cite{PearceLazard2010Design} --- 
targeted at notifying teachers of students' progress and state during
exploratory learning activities in the classroom, notifying the
teacher of students' attainment of key indicators, and aiming to
inform the teacher's own interventions in the class. This novelty of
our TA tools has presented a number of methodological challenges,
which we discuss in Section~\ref{sec:methodology}. 
In the last two years, several similar
initiatives have appeared, including the recent approach
of~\cite{Gutierrez12} ---inspired by early work
in~\cite{Yardi08}--- that focuses on 
informing teachers of students' progress and need for help 
in the context of computer programing labs.
% on progress measurement and
% whether students are stuck in the context of programming labs.
Earlier work (e.g.~\cite{Gueraud09}) focuses mostly on the statistics
 of the interaction (e.g.~how often did the student produce a certain
 kind of indicator). As we discuss in the present paper, our requirements
 analysis with teachers showed that immediate on-line feedback
 about students' current status and progress
 is more valuable to the teacher than simple statistics in supporting the
 `orchestration' of students' use of technologies in the classroom. 
% as it allows orchestrating
% the use of technologies in the classroom.\ednote{MM: I don't understnad
%   the comparison with the statistics --- maybe because I haven't read
%   the work --- also not sure we ever say again what we mean with
%   on-line feedback. Do you mean that the work of Gueraud09 is
%   off-line? SG: Yes.}

The trend towards teacher support is recently growing 
also in the learning analytics community
(see for example~\cite{Crespo12,Zaldivar12,Pardo12}) and there is high
synergetic potential between that work and the work reported here.
Other related initiatives include 
using Web log data generated by course management systems
(e.g.~WebCT) to help instructors become aware of students'
activities in distance learning classes~\cite{Mazza07}; post-analysis of
the system's data logs for helping teachers understand students'
behaviour in adaptive tutorials~\cite{BenAnim08}; 
and providing
awareness information to teachers so as to support their role as 
moderators of multiple e-discussions~\cite{Wichmann09} or class-wide
collaborative activities supported by hand-held devices~\cite{CortezNussbaum2009}. 
%
Particularly interesting is the work described 
in~\cite{Avouris08}, that uses tools to analyse CSCL 
synchronous interaction to help the teacher; their use of ``rules'' to
find specific landmarks in the interaction bears some
similarity to our detection of interaction indicators.  
%
However, none of this work
focuses on exploratory learning activities specifically. 

% Stefan Weinbrenner, Jan Engler, Astrid Wichmann, Ulrich Hoppe
% Monitoring and Analyzing Students' Systematic Behaviour by Agents -
% The SCY Pedagogical Agent Framework. Proceedings of the 5th European
% Conference on Technology Enhanced Learning (EC-TEL 2010), Barcelona,
% ES, 2010

% -- this I think is one of those mentioned at the top
% Stefan Weinbrenner, Jan Engler, Astrid Wichmann, Ulrich Hoppe
% Monitoring and Analyzing Students' Systematic Behaviour by Agents -
% The SCY Pedagogical Agent Framework. Proceedings of the 5th European
% Conference on Technology Enhanced Learning (EC-TEL 2010), Barcelona,
% ES, 2010


% Discussion here also of other related work on: 
% \begin{itemize}
% \item  interaction analysis
% \item  work on CSCL (even if we are not targetting collaboration here)
%   to support teachers 
% \item  work on teachers’ tools for moderating online discussions and
%   argumentation 
% \item  other tools for learning
% \item Argonaut's teacher tools (Astrid Wichman, Ulrich Hoppe)
% \end{itemize}


%%% Local Variables:
%%% mode: latex
%%% TeX-master: "main"