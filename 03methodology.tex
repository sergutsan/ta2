\section{Methodology}
\label{sec:methodology}

In our experience, teachers are not used to having access to tools
such as the MiGen’s TA tools, and their normal instinct is to walk
round the class to see how individual students are progressing and to
help them.  Indeed, in the first two years of the MiGen project,
before early versions of the TA tools were available, this was the
strategy that teachers adopted in helping students who were
undertaking tasks using the eXpresser.  As we discussed in Section 1,
this approach had significant limitations due to the inability of the
teacher to keep track of how the whole class of students were
progressing, each at their individual pace and using their own
construction approach to the task at hand. Because of teachers’
general lack of experience with such tools, it was not possible for us
to elicit from the outset a set of requirements for the TA tools from
our teacher collaborators on the project. Instead, it has been
necessary to adopt an agile, iterative methodology comprising
successive phases of prototyping, requirements elicitation,
development and evaluation in collaboration with the teachers. 

In particular, the development of the TA tools has been undertaken in
four phases during the course of the project, Phases A - D, which we
describe below. The number of teachers with whom we have been able to
collaborate closely during the course of the project is relatively
small due to practical considerations of the available staff resources
and the duration of the project. In particular, our Teacher Advisory
group comprised around 20 maths teachers and mathematics educators
from a broad spectrum of secondary schools in the greater London area,
who attended regular project team meetings and gave their input to the
development of the tools. However, the time that these teachers had
available to use early prototypes of the tools in their classrooms was
limited due to their need to deliver a tightly timetabled mathematics
curriculum, therefore collaboration with a core group of 4 teachers
played a prominent role, particularly in the early stages of the
research.
 
After a first version of the eXpresser and of the student feedback
provided by the eGeneraliser had been developed during the first 15
months of the project, we undertook a first phase of prototyping and
requirements elicitation for the TA tools working with our teacher
collaborators (Phase A, January 2009 - June 2010). Mockups and
prototypes of several possible visualisations were developed and
discussed in meetings of the Teacher Advisory group and in one-to-one
interviews with teachers who had already trialled the eXpresser and
eGeneraliser in their classrooms. The aim of these interactions was to
elicit teachers’ views about what information relating to students’
progress would be useful for them to have from the MiGen system as
students are working on eXpresser tasks, and how they would like this
information to be presented to them.
  
As an outcome of Phase A, two visualisations were co-developed with
the teachers, which subsequently evolved into the Classroom Dynamics
(CD) and Student Tracking (ST) tool (see [ECTEL’10paper] and Section 4
below). Also identified were a preliminary set of TI and TD indicators
to be monitored by the system as the students are working on the task
set and presented to the teacher by the ST tool.

The next phase of development of the TA tools involved several
classroom sessions trialling early prototypes of the ST tool (Phase B,
July 2010 - December 2010). Details of these early trials are given in
[IEEETLTPaper]. Briefly, the first version of the ST tool was used in
a classroom trial in July 2010. At that time, only event-based
indicators were supported by the system, and one of the major items of
feedback received from the teacher was that some of these indicators
were actually showing changes in the {\em state} of the student rather
than being single events, and should therefore be visualised as
vertical bars in the display that change colour when their status
changes. This feedback was incorporated into the development of the
next version of the ST tool. Another major item of feedback from the
teacher was that the set of indicators displayed by the ST tool should
be expanded to show also when and what prompts are being generated by
the system for each student. This feedback too was incorporated into
the next version of the ST tool.
 
In September 2010, a second classroom trial of the eXpresser and ST
tool was carried out with the same teacher using the new version of
the ST tool, and also with another of our teacher collaborators in a
different school. For both sessions, the ST tool was installed on the
teacher’s computer and we observed that the teachers afforded little
time consulting the tool, spending most of their time in their
familiar pattern of walking around the classroom to see what students
were doing and to help them. In post-lesson interviews, the two
teachers suggested two ways of alleviating this problem: installing
the Teacher tools on tablet PCs, which would then allow teachers to
view these tools as they are walking around the classroom; and
projecting the Teacher tools' display onto the whiteboard at the front
of the class, again allowing teachers to monitor the progress of
students as they are walking around the classroom. Both of these
approaches were adopted for subsequent classroom trials of the system,
in Phase D.
 
Another major item of feedback received from both teachers was that
the information shown by the ST tool was too detailed to be useful to
them during the lesson. However, they both felt it would be useful to
be able to track this level of detail for individual students after
the end of the lesson, for example for those students whose detailed
progress and achievements they need to check on before planning the
next lesson. As a result of this feedback, in Phase C we consulted
with pedagogical experts on deriving a subset of the most significant
indicators to be displayed by default in the ST tool. Teachers can
choose to `switch on' more indicators or `swich off' any of this
default set by means of an indicator-selection feature (see Section
4).

Following the Phase B classroom trials, a series of one-to-one
interviews were held with teachers so as to inform the further
development of the TA tools, and also to gain insight into how the
teacher will use such tools in practice in the classroom. As an
outcome of this, the need for a third tool was identified - the Goal
Achievements tool. Also, a set of Usage Scenarios for the whole suite
of CD, ST and GA tools were identified, which we present in Section
3.1 below.

The next phase of development of the TA tools involved formative
evaluation of the whole suite of tools with respect to the usage
scenarios identified from Phase B (Phase C, January 2011 - May
2011). This evaluation was undertaken firstly with a group of trainee
maths teachers on the Postgraduate Certificate in Education programme
at the Institute of Education (IoE), at the University of London, and
subsequently with a group of pedagogical experts in maths
education. We report on the design and outcomes of these formative
evaluation activities in Section 5.
 
The final phase of development of the TA tools (Phase D, June 2011 –
February 2012) involved summative evaluation undertaken in two
parts. In the first part, we conducted a classroom trial in which the
teacher was first introduced to the TA tools before the lesson, and
then used them during the lesson as students were working on a
generalisation task in the eXpresser. The teacher was asked questions
by a member of the research team during and after the lesson, with the
aim of evaluating the extent to which the TA tools meet the
requirements of the usage scenarios. After having used the TA tools in
one lesson, the same teacher was asked to undertake a similar lesson
the next day, but this time without referring to the TA tools as she
attempted to support the students working on a task in eXpresser. The
aim of this second session was to compare the difference in the
teacher’s experience compared to the first lesson in which she could
access the TA tools.
 
The second part of the summative evaluation of Phase D involved a
session held with a new cohort of trainee maths teachers on the
Postgraduate Certificate in Education programme at the IoE. After
introducing the eXpresser and TA tools to the participants and
presenting common ways in which these could be used in the classroom,
the participants were given access to the TA tools loaded with real
data from the recent classroom sessions undertaken in the first part
of the summative evaluation. They were given a questionnaire to probe
their views about the usage and effectiveness of the TA tools as well
as specific questions about the classroom status at specific times in
the lesson, which they had to answer in a limited amount of time -
simulating in this way the use of the tools during a classroom
session.
 
We report on the design and outcomes of these summative evaluation
activities in Section 6.

\subsection{Usage Scenarios}
\label{sec:usage-scenarios}

Following the early classroom trials held in Phase B, and the
subsequent interviews held with teachers, a set of key Usage Scenarios
for the combined set of ST, CD and GA tools were identified, which we
list as US1 – US8 below and which informed the design of the formative
and summative evaluations of the TA tools in Phases C and D.
 
US1: {\em Finding out which students need the teacher’s immediate
  help}. The teacher can consult the CD tool and see which students’
circles are coloured Red. For any such student, the teacher can click
on their circle to view their current model and rule, to provide some
context for the help the teacher may then give the student. The
teacher can also open up the ST tool to view the recent indicators
relating to the students’ actions, to provide additional context. If
there are more than one student coloured Red in the CD display, the
teacher may select to help first students who have achieved the fewest
task goals.
 
US2: {\em Finding out which students are progressing satisfactorily
  towards completing the task and which students may be in
  difficulty}. The teacher can consult the CD tool to see the goal
achievement numbers being displayed within the students’
circles. Students who have completed fewer numbers of goals may be in
difficulty. The teacher can click on their circle to view their
current model and rule, to provide additional context. The teacher can
also open up the GA tool to see which specific goals these students
have achieved. As in US1, the teacher can also open up the ST tool
display for a more detailed view of the students’ recent actions,
again providing additional context.

US3: {\em Finding out which students are currently disengaged from the
  task.} The teacher can consult the CD tool and see which students’
circles are currently coloured Amber. Looking at the number of goals
each of these students has achieved, if a student has not completed
the task goals, then she/he is likely to be currently disengaged from
the task and in need of encouragement from the teacher. If a student
has completed all the task goals, then she/he may need to be set
additional goals or a new task to work on while waiting for the rest
of the class to finish.
 
US4: {\em Identifying common conceptual and procedural difficulties
  students are facing in order to provide more explanation to the
  class as a whole.} Consulting the GA tool allows the teacher to see
which task goals students are having difficulty completing, so as to
inform additional explanation to the class.  Consulting the ST tool
allows the teacher to see if there are specific Red indicators showing
in many of the students’ columns, indicating particular procedural
difficulties that students may be facing and again informing the
provision of additional explanation to the class.

US6: {\em Finding out which students have achieved which task goals.}
The GA tool can be used to see which students have achieved which of
the task goals.
 
US7: {\em Providing appropriate support and guidance to individual
  students (i) during the lesson, and (ii) after the lesson.} This is
a more open-ended use case which can be undertaken during the lesson
using a combination of the tools as described for US1, US2, U3 above,
and after the lesson by using the GA tool to see which task goals an
individual student has not managed to achieve, the CD tool to view the
student’s final model and rule as produced by the end of the lesson,
and the ST tool to view the student’s detailed history of interactions
during the lesson.
 
US8: {\em Reflecting on the class’s achievements and planning the next
  lesson.} Again, this is a more open-ended use case which can be
undertaken by the teacher using a combination of the GA tool, to see
which task goals have been largely achieved by the class, the CD tool
to view selected students’ models and rules, and the ST tool to see a
historical record of how students tackled the task during the lesson.



%%% Local Variables:
%%% mode: latex
%%% TeX-master: "main"
%%% End:
