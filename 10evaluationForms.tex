
\section{Formative Evaluation in the Lab, Phase~C}
\label{sec:form-eval-with}

\subsection{Evaluation form}
\label{sec:evaluation-form}

There are the question from the questionnaire form that had to be
filled in by participants, relating to information displayed by the TA
tools using time-stop functionality: (i) 30 minutes into the lesson,
and (ii) 5 minutes before the end of the lesson. For each question or
teacher's task the evaluation subjects had to write down how they
would answer the question using the tools. They could also add their
comments, if any.

\begin{enumerate}
\item Finding out which students need your immediate help
\item Which students are progressing satisfactorily towards completing
  the task goals? Which students may be in difficulty?
\item Which students are currently disengaged from the task?
\item Would this be a time-point that you would give more explanation
  to the class as a whole? And if so what would you say based on
  information from the tools? 
\item Which students have finished the task?
\item Which task goals have most students achieved?
\end{enumerate}

\subsection{Further comments}
\label{sec:further-comments}

At the end of the session, they were given another questionnaire form
in which they had to answer three questions for every question or
teacher's task: (i) How would you find out without using the TA tools;
(i) with the TA tools, (ii-a) if the tools helped you achieve this
task, how did they do it, (ii-b) if they did not, what additional
information would help you?

\begin{enumerate}
\item Finding out which students need your immediate help.
\item Finding out which students are progressing satisfactorily on an
  eXpresser activity and which ones may be in difficulty
\item Finding out which students are disengaged during the activity
\item Identifying common conceptual and procedural difficulties
  students are facing in order to provide more explanation to the
  class as a whole.
\item Providing appropriate support and guidance to individual
  students during the lesson
\item Providing appropriate support and guidance to individual
  students after the lesson
\item Finding out which students have finished the task
\item Finding out which students have achieved specific task goals
\item Pairing students for collaborative tasks
\item Planning the next lesson using the MiGen system
\item Planning the next lesson without using the MiGen system
\end{enumerate}

\section{Summative Evaluation in the Classroom, Phase~D}
\label{sec:summ-eval-classr}

The following questions were posed to the teacher by a member of the
research team, firstly during the first lesson when she had access to
the TA tools and, again, during the second lesson when she did not
have access to the TA tools. The questions were asked at precise
points in time in both cases, as indicated (the lesson lasted 55
minutes). Each question relates to one or more of the Usage Scenarios
described in Section\ref{sec:usage-scenarios}, as indicated in square
brackets.

\begin{description}
\item[Question 1 (15 minutes into the lesson): ] (i)~which
  students are progressing satisfactorily in undertaking the task?
  [relates to Usage Scenario US2], (ii)~which students may be in
  difficulty?~[US2], and (iii)~which students need your immediate
  help?~[US1]
\item[Question 2 (25min): ] for the students that you have chosen to
  help, what guidance did you give each one?~[US7]
\item[Question 3 (35min): ] which students may be disengaged from the
  task? [US3] which students have achieved Task Goal~3?~[US6]
\item[Question 4 (45min): ] what common conceptual or procedural
  difficulties are students facing?~[US4], what explanation might you
  give to the whole class at this time?~[US4]
\item[Question 5 (end of lesson): ] Now that the lesson has finished,
  (i)~which students have finished the task?~[US5], (ii)~what 
  additional guidance might you give to any particular
  students?~[US7], and (iii)~what are your views about the class’s
  achievements in this lesson, plus how might you plan the next
  lesson?~[US8] 
\end{description}



%%% Local Variables:
%%% mode: latex
%%% TeX-master: "main"
%%% End:
