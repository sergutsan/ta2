%%
%% This is file `elsarticle-template-harv.tex',
%% generated with the docstrip utility.
%%
%% The original source files were:
%%
%% elsarticle.dtx  (with options: `harvtemplate')
%% 
%% Copyright 2007, 2008 Elsevier Ltd.
%% 
%% This file is part of the 'Elsarticle Bundle'.
%% -------------------------------------------
%% 
%% It may be distributed under the conditions of the LaTeX Project Public
%% License, either version 1.2 of this license or (at your option) any
%% later version.  The latest version of this license is in
%%    http://www.latex-project.org/lppl.txt
%% and version 1.2 or later is part of all distributions of LaTeX
%% version 1999/12/01 or later.
%% 
%% The list of all files belonging to the 'Elsarticle Bundle' is
%% given in the file `manifest.txt'.
%% 
%% Template article for Elsevier's document class `elsarticle'
%% with harvard style bibliographic references
%% SP 2008/03/01

%\documentclass[preprint,12pt]{elsarticle}

%% Use the option review to obtain double line spacing
\documentclass[authoryear,preprint,12pt]{elsarticle}

%% Use the options 1p,twocolumn; 3p; 3p,twocolumn; 5p; or 5p,twocolumn
%% for a journal layout:
%% \documentclass[final,1p,times]{elsarticle}
%% \documentclass[final,1p,times,twocolumn]{elsarticle}
%% \documentclass[final,3p,times]{elsarticle}
%% \documentclass[final,3p,times,twocolumn]{elsarticle}
%% \documentclass[final,5p,times]{elsarticle}
%% \documentclass[final,5p,times,twocolumn]{elsarticle}

%% if you use PostScript figures in your article
%% use the graphics package for simple commands
%% \usepackage{graphics}
%% or use the graphicx package for more complicated commands
%% \usepackage{graphicx}
%% or use the epsfig package if you prefer to use the old commands
%% \usepackage{epsfig}

%% The amssymb package provides various useful mathematical symbols
\usepackage{amssymb}
%% The amsthm package provides extended theorem environments
%% \usepackage{amsthm}

%% The lineno packages adds line numbers. Start line numbering with
%% \begin{linenumbers}, end it with \end{linenumbers}. Or switch it on
%% for the whole article with \linenumbers.
%% \usepackage{lineno}

%\usepackage[draft]{ednote}
%\usepackage{ednote}
\journal{Computers and Education}

\renewcommand{\cite}{\citep}
\newcommand{\citenop}[1]{\citeauthor{#1}~(\citeyear{#1})}

\begin{document}

\begin{frontmatter}

%% Title, authors and addresses

%% use the tnoteref command within \title for footnotes;
%% use the tnotetext command for theassociated footnote;
%% use the fnref command within \author or \address for footnotes;
%% use the fntext command for theassociated footnote;
%% use the corref command within \author for corresponding author footnotes;
%% use the cortext command for theassociated footnote;
%% use the ead command for the email address,
%% and the form \ead[url] for the home page:
%% \title{Title\tnoteref{label1}}
%% \tnotetext[label1]{}
%% \author{Name\corref{cor1}\fnref{label2}}
%% \ead{email address}
%% \ead[url]{home page}
%% \fntext[label2]{}
%% \cortext[cor1]{}
%% \address{Address\fnref{label3}}
%% \fntext[label3]{}

\title{Assisting teachers in an exploratory learning environment
  for algebraic generalisation} 

%Informing the Design of
%   Intelligent Support for ELE by \\ Iterative Communication Capacity
% Tapering}

%% use optional labels to link authors explicitly to addresses:
%% \author[label1,label2]{}
%% \address[label1]{}
%% \address[label2]{}

\author{Sergio Gutierrez-Santos \and Manolis Mavrikis \and Eirini
  Geraniou \and Alexandra Poulovassilis \\ 
  \{sergut,ap\}@dcs.bbk.ac.uk, \{m.mavrikis,e.geraniou\}@lkl.ac.uk}

\address{London Knowledge Lab} 

\begin{abstract}
We have designed and developed an intelligent exploratory learning
environment to support 11-14 year-old students in their learning of
algebraic generalisation. The system also provides tools to assist
teachers in monitoring students' activities and progress as they are
working on construction tasks in the classroom. This paper discusses
teachers’ requirements for such tools and describes our iterative
approach to designing, developing and evaluating them in collaboration
with teachers. We identify the major usage scenarios for the tools and
describe the tools themselves. We describe the design and outcomes of
the formative and summative evaluations that were undertaken with the
tools. We draw conclusions relating to the value of these tools to
teachers 
\end{abstract}

\begin{keyword}
     teacher support 
\sep iterative design 
\sep intelligent support 
\sep exploratory learning environments
%% keywords here, in the form: keyword \sep keyword
%% PACS codes here, in the form: \PACS code \sep code
%% MSC codes here, in the form: \MSC code \sep code
%% or \MSC[2008] code \sep code (2000 is the default)
\end{keyword}

\end{frontmatter}

%% \linenumbers

%for reviewing process
%\cleardoublepage

%% main text



%% The Appendices part is started with the command \appendix;
%% appendix sections are then done as normal sections
%% \appendix

%% \section{}
%% \label{}

%\begin{thebibliography}{00}

%% \bibitem{label}
%% Text of bibliographic item

%\bibitem{}

%\end{thebibliography}

\section*{Acknowledgements} 

The MiGen project is funded by the ESRC/EPSRC Teaching and
Learning Research Programme (Technology Enhanced Learning; Award no:
RES-139-25-0381). The authors would
like to thank the teachers and students who took part in the
wizard-of-Oz studies, and acknowledge the contribution of the MiGen team and
particularly Eirini Geraniou for the design of and
 participation in the studies, and Darren Pearce and Richard Noss for their comments in various versions of this paper. 

%\section*{Acknowledgements} The authors would like to acknowledge the
%rest of the members of the MiGen team and in particular Richard Noss, Darren
%Pearce and Eirini Geraniou for their comments and contibution to the
%paper, and Ken Kahn for the development of eXpresser. In addition, we
%would like to thank the teachers and students who took part in the studies.

\bibliographystyle{elsarticle-harv}
\bibliography{2009-04-28-group-6308,cae-cal09}
\end{document}

\endinput
%%
%% End of file `elsarticle-template-harv.tex'.


