\section{Conclusion}
\label{sec:conclusion}

In order to design MiGen’s Teacher Assistance (TA) tools, we have
collaborated with a number of teacher educators and maths teachers in
secondary schools in the UK. Initial prototypes of the TA Tools were
designed based on their input. We then followed an iterative design
process in order to identify the requirements that teachers regarded
as crucial in supporting their role in the classroom. Over the course
of the project, we conducted several one-to-one, small-scale, and
whole-classroom trials with our system in a number of secondary
schools in the UK with 11-14-year-old learners. We observed the use of
the TA tools by the teachers, especially their reactions and their
methods of incorporating the tools into their lessons. It was evident
that once our teacher collaborators had used the tools in their
classrooms, they were able to give us more informed feedback and
influence more clearly the subsequent development of the tools. We
have also conducted evaluations of the TA tools with the help of
groups of trainee maths teachers, using real students’ interaction
data collected from the school trials and the time-stop functionality
of the TA tools to simulate the classroom state at different times in
the lesson.

Working with teachers and pedagogical experts, we identified early on
a key set of interaction indicators that the TA tools needed to track
and the ST tool was developed to track this full set. However, during
early trials of the ST tool in the classroom, it became apparent that
this level of information was generally too detailed to be useful to
the teacher during the lesson. Following these experiences, a small
subset of indicators that were of most relevance to be tracked during
the lesson was identified, and became the default set displayed in the
ST tool. Also, the CD tool and the GA tool were co-developed with our
teacher collaborators, each of which is based on just a small number
of indicators, targeted at providing information about specific
aspects of the classroom state.

Based on the teacher’s feedback from early classroom trials, we
identified that providing a tablet PC where the TA tools were
installed helped the teacher to move around the class rather than
having to return back to their desk to view the tools, change the tool
being displayed, focus in on one aspect etc. In the lessons, we found
that teachers consulted regularly the Classroom Dynamics tool on their
tablet PC. With this tool, they decided early in the lesson to move
the circles representing each student to reflect the seating plan of
their class. As most teachers stated afterwards, this helped them use
the tool more effectively and locate quickly which students needed
help at any given moment. As soon as a circle turned red, the teacher
clicked on the circle to investigate the student’s current
construction and prepare their intervention. In one trial, the teacher
decided to also display the Classroom Dynamics tool on the Interactive
Whiteboard. This encouraged students to stay on task, as they were
aware of other students being able to view their progress. Teachers
also regularly viewed the Goals Achievement tool during the lesson, to
view students’ overall progress in terms of task completion. They
found this tool very useful when deciding which students to help based
on the task goals they had achieved, but also when designing
subsequent lessons.

Our future plans involve further research into providing more support
to teachers for the more complex usage scenarios relating to
identifying common difficulties that students are facing and
formulating appropriate guidance for students. We are currently in the
process of sharing the MiGen system with more teachers and
disseminating our results to a wider community. We will continue our
research to investigate further how such support for the teacher
influences the adoption of constructionist learning in the
classroom. Other future work involves investigating how the TA tools
could be adapted to support teachers using other exploratory learning
environments, such as the Metafora project’s …. and beyond that
virtual science labs, medical simulators, interactive environments for
notice programmers.



%%% Local Variables:
%%% mode: latex
%%% TeX-master: "main"
%%% End:
