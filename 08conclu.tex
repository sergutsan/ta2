\section{Conclusion}
\label{sec:conclusion}

In this paper, we have described the iterative development and evaluation of a suite
of Teacher Assistance (TA) tools that target an exploratory learning setting ---
specifically, the learning of algebraic generalisation --- and
that aim to help the teacher in enhacing her awareness of the classroom state, 
monitoring students' progress on the tasks set, and providing support to students. 
In order to design these tools and to identify key usage scenarios 
we have collaborated with a number of teacher educators and 
maths teachers in secondary schools in the UK. 

Over the course of the MiGen project, we have conducted several one-to-one, 
small-scale, and whole-classroom trials in a number of secondary
schools in the UK, with 11 to 14-year-old learners and their teachers. 
It was evident that once our teacher collaborators had used the TA tools in their
classrooms, they were able to give us more informed feedback and
influence more clearly the subsequent development of the tools. 

The development of time-stop functionality across all the TA tools 
allowed us to conduct evaluations of the TA tools with a far greater
number of teachers (specifically, trainee Maths teachers) 
than those who were able to participate in the classroom trials. 
The time-stop functionality allowed us to use real interaction data 
gathered from classroom trials and present that data to participants
via the TA tools `frozen' at particular moments in time,  
simulating in this way the experience of using the tools in a real classroom. 
The situation is of course not identical, as the evaluation participants
are not under simultaneous pressure from students asking for their help 
or trying to keep the lesson on track while they use the TA tools. 


% Working with teachers and pedagogical experts, we identified early on
% a key set of interaction indicators that the TA tools needed to track
% and the ST tool was developed to track this full set. However, during
% early trials of the ST tool in the classroom, it became apparent that
% this level of information was generally too detailed to be useful to
% the teacher during the lesson. Following these experiences, a small
% subset of indicators that were of most relevance to be tracked during
% the lesson was identified, and became the default set displayed in the
% ST tool. Also, the CD tool and the GA tool were co-developed with our
% teacher collaborators, each of which is based on just a small number
% of indicators, targeted at providing information about specific
% aspects of the classroom state.

Based on the teachers' feedback from early classroom trials, we
identified that installing the TA tools on a tablet PC 
helps the teacher to move around the classroom rather than
having to return back to their desk to view the tools, change the tool
being displayed, focus in on one aspect etc. 
Our TA tools have been tested and used by teachers on tablet PCs; it
would be straightforward to adapt them to an even smaller screen such
as that of a smartphone and this is an area of ongoing work.
In one early trial, the teacher decided to also display the 
Classroom Dynamics tool on the Interactive Whiteboard. The teacher
reported that this encouraged students to stay on task, as they were
aware of other students being able to view their progress. 
%
% In the classroom trials, we found that the teacher consulted regularly the 
% Classroom Dynamics tool on the tablet PC. 
% In this tool, they decided early in the lesson to move
% the circles representing each student to reflect the seating plan of
% their class. 
% As soon as a circle turned red, teachers
% clicked on the circle to investigate the student's current
% construction and rule and to prepare an appropriate intervention.
% Teachers also regularly viewed the Goals Achievement tool during the lesson, to
% view students' overall progress in terms of task completion. They
% found this tool very useful when deciding which students to help based
% on the task goals they had achieved, but also when designing
% subsequent lessons.
%
% One aspect that is worth mentioning is the effect that this kind of
% \emph{teacher awareness} has on the students. 
%
In the latest classroom trial, students reacted strongly when the teacher 
told them that she was able to observe from her tablet PC what they were 
doing on their computers. % (some of them quite vocally). 
It was apparent that the sensation of being monitored by the teacher 
led to better general behaviour and more focused work on the part of 
the students. 
We cannot tell without further empirical studies if this 
`better behaviour under vigilance' effect 
would last if the TA tools were used on a regular basis, or whether
students would soon return to their old habits. 
We believe that the effect could be a lasting one if students realised that their
actions would have consequences, e.g. if they did not finish
the task set in the lesson, the teacher would assign them finishing the task
as additional homework, or would take non-completion into account in assessing
their overall performance on this topic. 

We believe that visualisation and notification tools such as MiGen's TA tools 
that are developed specifically to support the teacher in an exploratory learning setting 
in the classroom are better able to provide a sense of awareness for the teacher 
than are general-purpose screen monitoring tools.
% ~\cite{monitor1,monitor2,monitor3,monitor4,monitor5}) 
Screen monitoring tools are not designed for continuous observation 
of the actions of many students during an entire lesson. 
Such tools would require a significant effort
on the part of the teacher to follow and analyse students' actions
comprising clicks, opening and closing of windows, etc. Moreover, such
tools require large screens to be really useful, which is not feasible
during a typical lesson, where the teacher generally needs to be able
to walk round the classroom helping students in addition to using the
tools to monitor the overall classroom state. 
% Although teachers
% usually have a dedicated desk-top computer and often a whiteboard in
% the classroom, our studies in classrooms have shown that teachers
% prefer to interact with the TA tools on a portable device that they
% can carry with them as they walk around the class, rather than being
% required to walk back to their desk at the front of the class in order
% to view a tool, change the tool selected for display, focus in on one aspect etc. 

Our future plans involve research into providing more targeted 
assistance to teachers for the more complex usage scenarios relating to
identifying common difficulties that students are facing and
formulating appropriate guidance for students. We are currently in the
process of sharing the MiGen system with more teachers and
disseminating our results to a wider community. We will continue 
to investigate how such assistance for the teacher
influences the adoption of exploratory learning in the classroom. 

Finally, it is important to note that the TA tools presented in this
paper are general in their design and that such tools could be used to
monitor the activities of students interacting with other exploratory
learning environments provided that the environment 
detects appropriate interaction indicators.
This would need to include as a minimum indicators relating to 
students' current activity status, 
waiting for help from the teacher, and goal
achievement status: 
these are the indicators that drive the CD and GA tool visualisations 
which we have found that, in practice, teachers consult most often during a lesson. 
Our future work therefore involves investigating how the TA tools
could be adapted to support teachers using other exploratory learning
environments targetting mathematics learning, e.g. in the Metafora (www.metafora-project.org/) 
and iTalk2Learn (www.italk2learn.eu/) EU-funded projects, 
and beyond that for virtual science labs, medical simulators, 
and interactive environments for novice programmers.



%%% Local Variables:
%%% mode: latex
%%% TeX-master: "main"
%%% End:
