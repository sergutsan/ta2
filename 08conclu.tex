\section{Conclusion}
\label{sec:conclusion}

In this paper, we have described the iterative development and evaluation of a suite
of Teacher Assistance (TA) tools that target an exploratory learning setting ---
specifically, the learning of algebraic generalisation.
%  --- and
% that aim to help the teacher in enhacing her awareness of the classroom state, 
% monitoring students' progress on the tasks set, and providing support to students. 
In order to design the tools and identify key usage scenarios 
we have collaborated with a number of teacher educators and 
maths teachers in secondary schools in the UK. 
Over the course of the MiGen project, we have conducted several one-to-one, 
small-scale, and whole-classroom trials in a number of 
schools, with 11 to 14-year-old learners and their teachers. 
% It was evident that once our teacher collaborators had used the TA tools in their
% classrooms, they were able to give us more informed feedback and
% influence more clearly the subsequent development of the tools. 

The results of the formative and summative evaluation sessions reported
in this paper show that participants were able to use the TA tools quickly and 
effectively to address usage scenarios US1, US2, US3, US5, US6, US8 ---
concerning teachers' awareness of the classroom state, students' progress 
on achieving the task goals, students in need of immediate help,
and reflection on the class' achievements ---
and that they appreciated the usefulness of the tools for these scenarios. 
There were some difficulties in using the tools for usage scenarios US4 and US7 ---
concerning identifying common problems
students are facing and formulating their guidance to students. 
A small number of the participants noted 
that the information displayed by the ST tool would 
allow teachers ``to identify the most common misconceptions which could then be
consolidated in the following lesson'', and this is a promising
starting point for further research. 

The development of time-stop functionality across all the TA tools 
allowed us to conduct evaluations of the tools with a far greater
number of teachers (specifically, trainee Maths teachers) 
than those who were able to participate in classroom trials. 
The time-stop functionality allowed us to use real interaction data 
from classroom trials and present it to participants
via the TA tools `frozen' at particular moments in time,  
simulating in this way the experience of using the tools in a real classroom. 
% The situation is of course not identical, as the evaluation participants
% are not under simultaneous pressure from students asking for their help 
% or trying to keep the lesson on track while they use the TA
% tools.\ednote{I think there is no point having the last two sentences
%  and this whole paragraph actually} 

%% Working with teachers and pedagogical experts, we identified early on
%% a key set of interaction indicators that the TA tools needed to track
%% and the ST tool was developed to track this full set. However, during
%% early trials of the ST tool in the classroom, it became apparent that
%% this level of information was generally too detailed to be useful to
%% the teacher during the lesson. Following these experiences, a small
%% subset of indicators that were of most relevance to be tracked during
%% the lesson was identified, and became the default set displayed in the
%% ST tool. Also, the CD tool and the GA tool were co-developed with our
%% teacher collaborators, each of which is based on just a small number
%% of indicators, targeted at providing information about specific
%% aspects of the classroom state.

Based on the teachers' feedback from early classroom trials, we
identified that installing the TA tools on a tablet PC 
helps the teacher to move around the classroom rather than
having to return back to their desk to interact with the tools.
% view the tools, change the tool
% being displayed, focus in on one aspect etc. 
% Our TA tools have been tested and used by teachers on tablet PCs; 
It would be straightforward to adapt the tools to an even smaller screen such
as that of a smartphone and this is an area of ongoing work.
% In one early trial, the teacher decided to also display the 
% Classroom Dynamics tool on the Interactive Whiteboard. The teacher
% reported that this encouraged students to stay on task, as they were
% aware of other students being able to view their progress. 
%%
%% In the classroom trials, we found that the teacher consulted regularly the 
%% Classroom Dynamics tool on the tablet PC. 
%% In this tool, they decided early in the lesson to move
%% the circles representing each student to reflect the seating plan of
%% their class. 
%% As soon as a circle turned red, teachers
%% clicked on the circle to investigate the student's current
%% construction and rule and to prepare an appropriate intervention.
%% Teachers also regularly viewed the Goals Achievement tool during the lesson, to
%% view students' overall progress in terms of task completion. They
%% found this tool very useful when deciding which students to help based
%% on the task goals they had achieved, but also when designing
%% subsequent lessons.
%%
%% One aspect that is worth mentioning is the effect that this kind of
%% \emph{teacher awareness} has on the students. 
%%


Our future plans involve research into providing more targeted 
assistance to teachers for the more complex usage scenarios relating to
identifying common difficulties that students are facing and
formulating appropriate guidance for students. We are currently in the
process of sharing the MiGen system with more teachers and
disseminating our results to a wider community. We will continue 
to investigate how such assistance for the teacher
influences the adoption of exploratory learning in the classroom. 

Finally, it is important to note that the TA tools presented in this
paper are general in their design and that such tools could be used to
monitor the activities of students interacting with other exploratory
learning environments provided that the environment 
detects appropriate interaction indicators.
This would need to include as a minimum indicators relating to 
students' current activity status, 
waiting for help from the teacher, and goal
achievement status: 
these are the indicators that drive the CD and GA tool visualisations 
which we have found that, in practice, teachers consult most often during a lesson. 
Our future work therefore involves investigating how the TA tools
could be adapted to support teachers using even more complex
exploratory learning environments (such as that of the Metafora
project (www.metafora-project.org)~\cite{Dragon13}, which focuses on
meta-learning) and beyond that for virtual science labs, medical simulators, 
and interactive environments for novice programmers.

%%% Local Variables:
%%% mode: latex
%%% TeX-master: "main"
%%% End:
